\documentclass[10pt]{article}
\usepackage[utf8]{inputenc}
\usepackage{amscd}
\usepackage{amsmath}
\usepackage{amssymb}
\usepackage{amsthm}
\usepackage{listings}
\usepackage{enumerate}

\textwidth=15cm \textheight=22cm \topmargin=0.5cm \oddsidemargin=0.5cm \evensidemargin=0.5cm

\newcommand{\sk}{\vskip 10mm}
\newcommand{\bb}[1]{\mathbb{#1}}
\newcommand{\ra}{\rightarrow}

\theoremstyle{plain}
\newtheorem{problem}{Problem}
\newtheorem{lemma}{Lemma}[problem]

\theoremstyle{remark}
\newtheorem{tpart}{}[problem]
\newtheorem*{ppart}{}

\begin{document}

\begin{problem}
  Prove the Brouwer fixed point theorem for maps $f:D^n\rightarrow D^n$
  by applying degree theory to the map $S^n\rightarrow S^n$ that sends
  both the northern and southern hemispheres of $S^n$ to the
  southern hemisphere via $f$. [This was Brouwer's original proof.]
\end{problem}

\begin{proof}
  Suppose that we have a map $f:D^n\rightarrow D^n$ with no fixed points.
  Then define a map $g:S^n\rightarrow S^n$ by identifying the northern
  hemisphere as the destination of $f$ via
  \[
    g(x) =
    \left\{
      \begin{array}{ll}
        f(x) & x\in \text{northern hemisphere}\\
        f(-x) & \text{otherwise}               
      \end{array}
    \right.
  \]
  Then $g$ has degree 0 since it is not surjective and $g$
  has degree $(-1)^{n+1}$ since it has no fixed point from Hatcher pg. 134.
  This is a contradiction.
\end{proof}

\sk

\begin{problem}
  Given a map $f:S^{2n}\rightarrow S^{2n}$, show that there is some point
  $x\in S^{2n}$ with either $f(x)=x$ or $f(x)=-x$. Deduce that
  every map $\bb{R}P^{2n}\rightarrow \bb{R}P^{2n}$ has a fixed point.
  Construct maps $\bb{R}P^{2n-1}\rightarrow\bb{R}P^{2n-1}$ without fixed points
  from linear transformations $\bb{R}^{2n}\rightarrow\bb{R}^{2n}$ without
  eigenvectors.
\end{problem}

\begin{proof}
  Suppose otherwise. Then $f$ has no fixed point and as such
  has degree $(-1)^{2n+1}=-1$. Moreover the map $-f=f\circ A$ also
  has no fixed point and as such has degree $(-1)^{2n+1}=-1$.
  However due to how degree multiplies under composition we also
  have that $-\deg f=\deg -f$ which is a contradiction.
\end{proof}

\sk

\begin{problem}
  Let $f:S^n\rightarrow S^n$ be a map of degree zero. Show that there exist
  points $x,y\in S^n$ with $f(x)=x$ and $f(y)=-y$. Use this to show
  that if $F$ is a continuous vector field defined on the unit
  ball $D^n$ in $\bb{R}^n$ such that $F(x)\neq 0$ for all $x$, then
  there exists a point on $\partial D^n$ where $F$ points radially outward
  and another point on $\partial D^n$ where $F$ points radially inward.
\end{problem}

\begin{proof}
  
\end{proof}

%%%%%%%%%%%%%%%%%%%%%%%%%%%%%%%%%%%%%%%%%%%%%%%%%%%%%%%%%%%%%%%%%%%%%%%%%%%%%
\end{document}
