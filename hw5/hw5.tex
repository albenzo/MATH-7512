\documentclass[10pt]{article}
\usepackage[utf8]{inputenc}
\usepackage{amscd}
\usepackage{amsmath}
\usepackage{amssymb}
\usepackage{amsthm}
\usepackage{listings}
\usepackage{enumerate}
\usepackage[all,cmtip]{xy}

\textwidth=15cm \textheight=22cm \topmargin=0.5cm \oddsidemargin=0.5cm \evensidemargin=0.5cm

\newcommand{\sk}{\vskip 10mm}
\newcommand{\bb}[1]{\mathbb{#1}}
\newcommand{\ra}{\rightarrow}

\theoremstyle{plain}
\newtheorem{problem}{Problem}
\newtheorem{lemma}{Lemma}[problem]

\theoremstyle{remark}
\newtheorem{tpart}{}[problem]
\newtheorem*{ppart}{}

\begin{document}

\begin{problem}
  Prove the Brouwer fixed point theorem for maps $f:D^n\rightarrow D^n$
  by applying degree theory to the map $S^n\rightarrow S^n$ that sends
  both the northern and southern hemispheres of $S^n$ to the
  southern hemisphere via $f$. [This was Brouwer's original proof.]
\end{problem}

\begin{proof}
  Suppose that we have a map $f:D^n\rightarrow D^n$ with no fixed points.
  Then define a map $g:S^n\rightarrow S^n$ by identifying the northern
  hemisphere as the destination of $f$ via
  \[
    g(x) =
    \left\{
      \begin{array}{ll}
        f(x) & x\in \text{northern hemisphere}\\
        f(-x) & \text{otherwise}               
      \end{array}
    \right.
  \]
  Then $g$ has degree 0 since it is not surjective and $g$
  has degree $(-1)^{n+1}$ since it has no fixed point from Hatcher pg. 134.
  This is a contradiction.
\end{proof}

\sk

\begin{problem}
  Given a map $f:S^{2n}\rightarrow S^{2n}$, show that there is some point
  $x\in S^{2n}$ with either $f(x)=x$ or $f(x)=-x$. Deduce that
  every map $\bb{R}P^{2n}\rightarrow \bb{R}P^{2n}$ has a fixed point.
  Construct maps $\bb{R}P^{2n-1}\rightarrow\bb{R}P^{2n-1}$ without fixed points
  from linear transformations $\bb{R}^{2n}\rightarrow\bb{R}^{2n}$ without
  eigenvectors.
\end{problem}

\begin{proof}
  Suppose otherwise. Then $f$ has no fixed point and as such
  has degree $(-1)^{2n+1}=-1$. Moreover the map $-f=f\circ A$ also
  has no fixed point and as such has degree $(-1)^{2n+1}=-1$.
  However due to how degree multiplies under composition we also
  have that $-\deg f=\deg -f$ which is a contradiction.

  \sk

  For the second portion suppose that we have a map
  $g:\bb{R}P^{2n}\rightarrow\bb{R}^{2n}$ and consider its lift
  $\widetilde{g}:S^{2n}\rightarrow S^{2n}$. Then we have the commutative
  diagram
  \[
    \xymatrix{
      S^{2n} \ar[r]^{\widetilde{g}} & S^{2n} \ar[d]^q\\
      \bb{R}P^{2n} \ar[u]^i \ar[r]^g & \bb{R}P^{2n}
    }
  \]
  where $i$ is inclusion and $q$ is a quotient map that agrees
  with the inclusion. Since $\widetilde{g}$ must respect
  equivalence classes we have that $\widetilde{g}(x)=\widetilde{g}(-x)$.
  Moreover from the previous part we have that there is an
  $x\in S^{2n}$ such that $\widetilde{g}(x)=\widetilde{g}(-x)$ is either
  $x$ or $-x$. In either case there is a corresponding element
  $\bar{x}\in\bb{R}P^{2n}$ such that $i(\bar{x})$ is $\pm x$ and $q(\pm x)=\bar{x}$.
  By commutativity this implies that $g(\bar{x})=\bar{x}$ which shows
  that $g$ indeed has a fixed point.  
\end{proof}

An example of a map from $\bb{R}^{2n-1}$ to $\bb{R}P^{2n-1}$ is
the map
\[
  (a_1,a_2,\ldots,a_n,a_{n+1})\mapsto (-a_2,a_2,\ldots,-a_{n+1},a_n)
\]
where $(a_1,\ldots, a_{n+1})$ is a point in $\bb{R}P^n$ such
that not all $a_i$ are zero. This map is well defined since
the dimension is odd and will not have a fixed point.

\sk

\begin{problem}
  Let $f:S^n\rightarrow S^n$ be a map of degree zero. Show that there exist
  points $x,y\in S^n$ with $f(x)=x$ and $f(y)=-y$. Use this to show
  that if $F$ is a continuous vector field defined on the unit
  ball $D^n$ in $\bb{R}^n$ such that $F(x)\neq 0$ for all $x$, then
  there exists a point on $\partial D^n$ where $F$ points radially outward
  and another point on $\partial D^n$ where $F$ points radially inward.
\end{problem}

\begin{proof}
  Suppose that $f$ had no fixed point. Then its degree would have to be
  $(-1)^{n+1}$ which is a contradiction. If $-f$ had no fixed point then
  $\deg -f=(-1)^{n+1}$ which is also a contradiction. Therefore if $f$
  has degree zero it must have a fixed point and so does $-f$.

  Let $F:D^n\rightarrow\bb{R}^n$ be a nonzero vector field on $D^n$. Then define
  $\bar{F}:D^n\rightarrow S^{n-1}$ by normalizing
  \[
    \bar{F}(x)=\frac{F(x)}{|F(x)|}
  \]
  Then the restriction of $\bar{F}$ to $\partial D^n$ is a map from $S^{n-1}$
  to $S^{n-1}$ that has degree zero since it extends to a disc. Thus
  there exist points $x,y\in \partial D^n$ such that $\bar{F}(x)=x$ and $\bar{F}(y)=-y$
  corresponding to a vector $F(x)$ and $F(y)$ pointing radially inward and
  radially outward on the boundary of $D^n$ respectively.
\end{proof}

%%%%%%%%%%%%%%%%%%%%%%%%%%%%%%%%%%%%%%%%%%%%%%%%%%%%%%%%%%%%%%%%%%%%%%%%%%%%%
\end{document}
