\documentclass[10pt]{article}
\usepackage[utf8]{inputenc}
\usepackage{amscd}
\usepackage{amsmath}
\usepackage{amssymb}
\usepackage{amsthm}
\usepackage{listings}
\usepackage{enumerate}
\usepackage[all,cmtip]{xy}

\textwidth=15cm \textheight=22cm \topmargin=0.5cm \oddsidemargin=0.5cm \evensidemargin=0.5cm

\newcommand{\sk}{\vskip 10mm}
\newcommand{\bb}[1]{\mathbb{#1}}
\newcommand{\ra}{\rightarrow}
\newcommand{\rh}{\widetilde{h}}

\theoremstyle{plain}
\newtheorem{problem}{Problem}
\newtheorem{lemma}{Lemma}[problem]

\theoremstyle{remark}
\newtheorem{tpart}{}[problem]
\newtheorem*{ppart}{}

\begin{document}

\begin{problem}[2.2.36]
  Show that $H_i(X\times S^n)\cong H_i(X)\oplus H_{i-n}(X)$ for all $i$ and $n$,
  where $H_i=0$ for $i<0$ be definition. Namely, show
  $H_i(X\times S^n)\cong H_i(X)\oplus H_i(X\times S^n,X\times\{x_0\})$ and
  $H_i(X\times S^n,X\times\{x_0\})\cong H_{i-1}(X\times S^{n-1},X\times\{x_0\})$
  [For the latter isomorphism the relative Mayer-Vietoris sequence yields
  an easy proof].
\end{problem}

\begin{proof}
  
\end{proof}

\sk

\begin{problem}[2.3.3]
  Show that if $\rh$ is a reduced homology theory then $\rh_n(point)=0$ for all
  $n$. Deduce that there are suspension isomorphisms $\rh_n(X)\cong\rh_{n+1}(SX)$
  for all $n$.
\end{problem}

\begin{proof}
  First note that if we wedge the point with itself that we get a point back.
  Then using the wedge axiom we have $\rh_i(\bigvee_1^2*=*)$ is isomorphic to
  $\rh_i(*)\oplus\rh_i(*)$. The only way this could occur would be if $\rh_i(*)\cong 0$
  for all $i$.

  For the suspension use the Mayer-Vietoris sequence along with the
  fact that the cone of a space is contractable to get
  \[
    \xymatrix{
      \cdots \ar[r] & \rh_{i+1}(CX)\oplus \rh_{i+1}(CX) \ar[r] & \rh_{i+1}(SX) \ar[r] & \rh_i(X) \ar[r] & \rh_i(CX)\oplus \rh_i(CX) \ar[r] & \cdots
    }
  \]
  Which is equivalent to
  \[
    \xymatrix{
      0 \ar[r] & \rh_{i+1}(SX) \ar[r] & \rh_i(X) \ar[r] & 0
      }
  \]
  This show that $\rh_i(X)\cong\rh_{i+1}(SX)$ for all $i$.
\end{proof}

\sk

\begin{problem}[2.B.3]
  Let $(D,S)\subset (D^n,S^{n-1})$ be a pair of subspaces homeomorphic to
  $(D^k,S^{k-1})$, with $D\cap S^{n-1}=S$. Show that the inclusion
  $S^{n-1}-S\hookrightarrow D^n-D$ induces an isomorphism on homology. [Glue two
  copies of $(D^n,D)$ to the two ends of $(S^{n-1}\times I,S\times I)$ to
  produce a $k$-sphere in $S^n$ and look at the Mayer-Vietoris sequence for the
  complement of this $k$-sphere.]
\end{problem}

\begin{proof}
  As described in the hint. Construct a $S^n\setminus S^k$ and decompose it
  as $A= D^n\setminus D$ and $B=S^n\setminus D$. Then $A\cap B=S^{n-1}\setminus S$. By
  Prop 2B.1 we have that $B$ is null-homologous. Then from the Mayer-Vietoris
  sequence we have
  \[
    \xymatrix{
      \cdots \ar[r] & H_i(S^{n-1}\setminus S) \ar[r] & H_i(D^n\setminus D) \ar[r] & H_i(S^n\setminus S^k) \ar[r] & \cdots
    }
  \]
  Where the map between $H_i(S^{n-1}\setminus S)$ and $H_i(D^n\setminus D)$ is inclusion. Then
  Using 2B.1 again we know that $S^n\setminus S^k$ will have zero homology everywhere
  but for $i=n-k-1$. This induces isomorphisms for $i>n-k-1$ and $i<n-k-2$.

  Finally for $i=n-k-1,n-k-2$ we have 
\end{proof}

%%%%%%%%%%%%%%%%%%%%%%%%%%%%%%%%%%%%%%%%%%%%%%%%%%%%%%%%%%%%%%%%%%%%%%%%%%%%%
\end{document}
