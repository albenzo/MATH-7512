\documentclass[10pt]{article}
\usepackage[utf8]{inputenc}
\usepackage{amscd}
\usepackage{amsmath}
\usepackage{amssymb}
\usepackage{amsthm}
\usepackage{listings}
\usepackage{enumerate}

\textwidth=15cm \textheight=22cm \topmargin=0.5cm \oddsidemargin=0.5cm \evensidemargin=0.5cm

\newcommand{\sk}{\vskip 10mm}
\newcommand{\bb}[1]{\mathbb{#1}}
\newcommand{\ra}{\rightarrow}

\theoremstyle{plain}
\newtheorem{problem}{Problem}
\newtheorem{lemma}{Lemma}[problem]

\theoremstyle{remark}
\newtheorem{tpart}{}[problem]
\newtheorem*{ppart}{}

\begin{document}

\begin{problem}[22]
  Prove by induction on dimension the following facts about the
  homology of a finite-dimensional CW complex $X$, using the observation
  that $X^n/X^{n-1}$ is a wedge sum of $n$-spheres.
  \begin{enumerate}
  \item[(a)] If $X$ has dimension $n$ then $H_i(X)=0$ for $i>n$ and
    $H_n(X)$ is free.
  \item[(b)] $H_n(X)$ is free with basis in bijective correspondence
    with the $n$-cells if there are no cells of dimension $n-1$ or $n+1$.
  \item[(c)]  If $X$ has $k$ $n$-cells then $H_n(X)$ is generated by
    at most $k$ elements.
  \end{enumerate}
\end{problem}

\begin{proof}
  
\end{proof}

\sk

\begin{problem}[26]
  Show that $H_1(X,A)$ is not isomorphic to $\widetilde{H}_1(X/A)$ if
  $X=[0,1]$ and $A$ is the sequence $1,\frac{1}{2},\frac{1}{3},\ldots$
  together with its limit $0$. [See Example 1.25.]
\end{problem}

\begin{proof}
  
\end{proof}

\sk

\begin{problem}[27]
  Let $f:(X,A)(Y,B)$ be a map that both $f:X\rightarrow Y$ and the restriction
  $f:A\rightarrow B$ are homotopy equivalences.
  \begin{enumerate}
  \item[(a)] Show that $f_*:H_n(X,A)\rightarrow H_n(Y,B)$ is an
    isomorphism for all $n$.
  \item[(b)] For the inclusion $f:(D^n,S^{n-1})\rightarrow(D^n,D^n-\{0\})$, show
    that $f$ is not a homotopy equivalence of pairs \textemdash there
    is no $g:(D^n,D^n-\{0\})\rightarrow(D^n,S^{n-1})$ such that $fg$
    and $gf$ are homotopic to the identity through maps of pairs.
    [Observe that a homotopy equivalence of pairs $(X,A)\rightarrow(Y,B)$
    is also a homotopy equivalence for the pairs obtained by
    replacing $A$ and $B$ by their closures.]
  \end{enumerate}
\end{problem}

\begin{proof}
  
\end{proof}

%%%%%%%%%%%%%%%%%%%%%%%%%%%%%%%%%%%%%%%%%%%%%%%%%%%%%%%%%%%%%%%%%%%%%%%%%%%%%
\end{document}
