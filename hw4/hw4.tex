\documentclass[10pt]{article}
\usepackage[utf8]{inputenc}
\usepackage{amscd}
\usepackage{amsmath}
\usepackage{amssymb}
\usepackage{amsthm}
\usepackage{listings}
\usepackage{enumerate}
\usepackage[all,cmtip]{xy}

\textwidth=15cm \textheight=22cm \topmargin=0.5cm \oddsidemargin=0.5cm \evensidemargin=0.5cm

\newcommand{\sk}{\vskip 10mm}
\newcommand{\bb}[1]{\mathbb{#1}}
\newcommand{\ra}{\rightarrow}
\newcommand{\wt}[1]{\widetilde{#1}}

\theoremstyle{plain}
\newtheorem{problem}{Problem}
\newtheorem{lemma}{Lemma}[problem]

\theoremstyle{remark}
\newtheorem{tpart}{}[problem]
\newtheorem*{ppart}{}

\begin{document}

\begin{problem}[22]
  Prove by induction on dimension the following facts about the
  homology of a finite-dimensional CW complex $X$, using the observation
  that $X^n/X^{n-1}$ is a wedge sum of $n$-spheres.
  \begin{enumerate}
  \item[(a)] If $X$ has dimension $n$ then $H_i(X)=0$ for $i>n$ and
    $H_n(X)$ is free.
  \item[(b)] $H_n(X)$ is free with basis in bijective correspondence
    with the $n$-cells if there are no cells of dimension $n-1$ or $n+1$.
  \item[(c)]  If $X$ has $k$ $n$-cells then $H_n(X)$ is generated by
    at most $k$ elements.
  \end{enumerate}
\end{problem}

\[
  \cdots \rightarrow \wt{H}_k(X^{n-1})\rightarrow \wt{H}_k(X^n) \rightarrow \wt{H}_k(X^n/X^{n-1}) \rightarrow \wt{H}_{k-1}(X^{n-1}) \rightarrow \wt{H}_{k-1}(X^n) \rightarrow \wt{H}_{k-1}(X^n/X^{n-1}) \rightarrow \cdots
\]

\begin{proof}
  \begin{enumerate}
  \item[(a)]
    Suppose that $X$ has dimension zero. Then $X$ consists of only $0$-cells
    and as such $C_i(X)=0$ for $i>0$ implying that $H_i(X)=0$ for $i>0$. From
    Hatcher we have that $H_0(X)=\bigoplus_{x\in X}\bb{Z}$.

    Now assume that for $Y$ with dimension $j<n$ that $H_i(Y)=0$ for $i>j$ and
    that $H_j(Y)$ is free. Suppose that $X$ has dimension $n$. Then from Hatcher
    we have the long exact sequence
    \[
      \cdots \rightarrow \wt{H}_k(X^{n-1})\rightarrow \wt{H}_k(X^n) \rightarrow \wt{H}_k(X^n/X^{n-1}) \rightarrow \wt{H}_{k-1}(X^{n-1}) \rightarrow \wt{H}_{k-1}(X^n) \rightarrow \wt{H}_{k-1}(X^n/X^{n-1}) \rightarrow \cdots
    \]

    If $k>n$ then this gives us the exact sequence
    \[
      0\rightarrow \wt{H}_k(X^n)\rightarrow 0
    \]
    as $\wt{H}_k(X^{n-1})$ will be zero by induction and $\wt{H}_k(X^n/x^{n-1})$ will
    be zero since $n$-spheres only have nonzero reduced homology in dimension $n$.

    Otherwise if $k=n$ we get the exact sequence
    \[
      0\rightarrow \wt{H}_n(X^n)\rightarrow \bigoplus_{n\text{ simplices}}\bb{Z}\rightarrow\cdots
    \]
    The fact that $\wt{H}_n(X^{n-1})$ is zero by induction forces the above map
    to be injective. It then follows that $\wt{H}_n(X^n)$ is isomorphic to the
    subgroup of a free abelian group and as such is free as well. Since reduced
    homology is identical to homology when the dimension is greater than zero
    this completes the proof.
    
  \item[(b)] Let $n=0$. From Hatcher we have that $H_0(X)$ is free abelian
    with a generator for each 0-cell.

    Now assume that for dimension $j<n$ if there are no $j-1$ or $j+1$ simplices
    that $H_j(X)$ is free with basis in bijective correspondence with the $j$-cells.
    Then let $X$ be a finite-dimensional CW complex with no $n$-simplices. Then
    via the same exact sequence from above we get
    \[
      \wt{H}_n(X^{n-1})\cong 0\rightarrow\wt{H}_n(X^n)\rightarrow\wt{H}_n(X^n/X^{n-1})\rightarrow\wt{H}_{n-1}(X^{n-1}\cong 0)
    \]
    Which induces an isomorphism between the remaining groups. Since
    $\wt{H}_n(X^n/X^{n-1})$ has a generator for each $n$ simplex it follows
    that $\wt{H}_n(X^n)$ does as well.

    Next suppose that there are no $n+1$ simplices.
  \item[(c)]
  \end{enumerate}
\end{proof}

\sk

\begin{problem}[26]
  Show that $H_1(X,A)$ is not isomorphic to $\widetilde{H}_1(X/A)$ if
  $X=[0,1]$ and $A$ is the sequence $1,\frac{1}{2},\frac{1}{3},\ldots$
  together with its limit $0$. [See Example 1.25.]
\end{problem}

\begin{proof}
  We can see that $H_1(X,A)\cong\bigoplus_1^\infty\bb{Z}$ from the short exact sequence
  \[
    \xymatrix{
      0\ar[r] & H_1(X,A) \ar[r] & H_0(A)\cong \bigoplus_0^\infty\bb{Z} \ar[r] & H_0(X)\cong \bb{Z} \ar[r] & 0
    }
  \]
  as this short exact sequence enforces that $H_1(X,A)\oplus \bb{Z}\cong \bigoplus_0^\infty\bb{Z}$. Thus
  $H_1(X,A)$ has a countable number elements.

  However if we examine the space $X/A$ we can see that it is homeomorphic to the
  Hawaiian earring. Moreover there are a countable number of one simplices and
  a single zero simplex. Thus $\widetilde{H}_1(X/A)$ is not finitely generated
  and as such by Hatcher (49) we have that $\widetilde{H}_1(X/A)$ is uncountable.

  Therefore $H_1(X,A)$ and $\widetilde{H}_1(X/A)$ are not isomorphic.
\end{proof}

\sk

\begin{problem}[27]
  Let $f:(X,A)\rightarrow(Y,B)$ be a map that both $f:X\rightarrow Y$ and the restriction
  $f:A\rightarrow B$ are homotopy equivalences.
  \begin{enumerate}
  \item[(a)] Show that $f_*:H_n(X,A)\rightarrow H_n(Y,B)$ is an
    isomorphism for all $n$.
  \item[(b)] For the inclusion $f:(D^n,S^{n-1})\rightarrow(D^n,D^n-\{0\})$, show
    that $f$ is not a homotopy equivalence of pairs \textemdash there
    is no $g:(D^n,D^n-\{0\})\rightarrow(D^n,S^{n-1})$ such that $fg$
    and $gf$ are homotopic to the identity through maps of pairs.
    [Observe that a homotopy equivalence of pairs $(X,A)\rightarrow(Y,B)$
    is also a homotopy equivalence for the pairs obtained by
    replacing $A$ and $B$ by their closures.]
  \end{enumerate}
\end{problem}

\begin{proof}
  \begin{enumerate}
  \item[(a)] From Hatcher we have the long exact sequences with morphisms
    between them like so:
    \[
      \xymatrix{
        \cdots \ar[r] & H_n(A) \ar[r] \ar[d]^{f_*} & H_n(X) \ar[r] \ar[d]^{f_*} & H_n(X,A) \ar[r] \ar[d]^{f_*} & H_{n-1}(A) \ar[r] \ar[d]^{f_*} & H_{n-1}(X) \ar[r] \ar[d]^{f_*} & \cdots \\
        \cdots \ar[r] & H_n(B) \ar[r] & H_n(Y) \ar[r] & H_n(Y,B) \ar[r] & H_{n-1}(B) \ar[r] & H_{n-1}(Y) \ar[r] & \cdots
      }
    \]

    Since $f$ is a homotopy equivalence for $A$ and $X$ the left 2 and the right 2 $f_*$s
    are isomorphisms. Thus by the five lemma the center map $f_*$ is an
    isomorphism.
  \item[(b)] Suppose otherwise. Then we have a map $g:(D^n,D^n\setminus\{0\})\rightarrow(D^n,S^{n-1})$
    such that $g$ is a homotopy equivalence. Then since $g:D^n\setminus\{0\}\rightarrow S^{n-1}$
    is continuous it will force $g:D^n\rightarrow S^{n-1}$ to send $0$ to $S^{n-1}$ as
    well as it is continuous. Since $g$ extends is then a map from $D^n$ that
    extends to a disc it is null and therefore the zero map on homology.

    However for $k=n-1$ we have a map
    \[
      H_{n-1}(D^n\setminus\{0\})\cong \bb{Z}\rightarrow H_n(S^{n-1}) \cong Z
    \]
    which is both the zero map and an isomorphism since it is a homotopy
    equivalence. This is a contradiction.

    Therefore there the inclusion map is not a homotopy equivalence of
    pairs for $(D^n,S^{n-1})$ and $(D^n,D^n\setminus\{0\})$.
  \end{enumerate}
\end{proof}

%%%%%%%%%%%%%%%%%%%%%%%%%%%%%%%%%%%%%%%%%%%%%%%%%%%%%%%%%%%%%%%%%%%%%%%%%%%%%
\end{document}
