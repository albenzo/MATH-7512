\documentclass[10pt]{article}
\usepackage[utf8]{inputenc}
\usepackage{amscd}
\usepackage{amsmath}
\usepackage{amssymb}
\usepackage{amsthm}
\usepackage{listings}
\usepackage{enumerate}
\usepackage[all,cmtip]{xy}

\textwidth=15cm \textheight=22cm \topmargin=0.5cm \oddsidemargin=0.5cm \evensidemargin=0.5cm

\newcommand{\sk}{\vskip 10mm}
\newcommand{\bb}[1]{\mathbb{#1}}

\theoremstyle{plain}
\newtheorem{problem}{Problem}
\newtheorem{lemma}{Lemma}[problem]

\theoremstyle{remark}
\newtheorem{tpart}{}[problem]
\newtheorem*{ppart}{}

\begin{document}

\begin{problem}[12]
  Show that the quotient map $S^1\times S^1\rightarrow S^2$ collapsing the
  subspace $S^1\vee S^1$ to a point is not nullhomotopic
  by showing that it induces an isomorphism on $H_2$. On the
  other hand, show via covering spaces that any map
  $S^2\rightarrow S^1\times S^1$ is nullhomotopic.
\end{problem}

\begin{proof}
  Consider the long exact sequence in reduced homology
  \[
    \xymatrix{
      \cdots \ar[r] & \widetilde{H}_2(S^1\vee S^1) \ar[r] & \widetilde{H}_2(T^2) \ar[r] & \widetilde{H}_2(S^2) \ar[r] & \widetilde{H}_1(S^1\vee S^1) \ar[r] & \widetilde{H}_1(T^2) \ar[r] & \cdots
    }
  \]

  First note that $H_2(S^1\vee S^1)= 0$ since there are no two cells which forces
  the map induced by the quotient to be injective.
  In addition the map from $\widetilde{H}_1(S^1\vee S^1)$ to $\widetilde{H}_1(T^2)$
  is an isomorphism. Thus the kernel of the boundary map is zero making
  the map induced by the quotient surjective.

  Therefore the map induced by the quotient is an isomorphism between nonzero groups
  and as such cannot be null.
  
  \sk
  
  Let $f$ be a map from $S^2\rightarrow T^2$. Then define $\bar{f}:I^2\rightarrow T^2$
  by the usual quotient of $I^2$ to $S^2$ by identifying the boundary.
  However $\bar{f}$ is a homotopy and as such it has a lift
  $\widetilde{f}:I^2\rightarrow \bb{R}^2$. However since $\bar{f}(\partial I^2)$ is
  a single point and the preimage of any point under the covering
  map is discrete it must be that $\widetilde{f}(\partial I^2)$ also is a
  single point. Thus $\widetilde{f}$ is can be identified with a map
  from $S^2$ to $\bb{R}^2$ such that $\widetilde{f}$ composed with the
  covering map is $f$. However since $f$ factors through a contractible
  space it follows that $f$ is null.
\end{proof}

\sk

\begin{problem}[14]
  A map $f:S^n\rightarrow S^n$ satisfying $f(x)=f(-x)$ for all $x$ is called
  an even map. Show that an even map $S^n\rightarrow S^n$ must have even
  degree, and that the degree must in fact be zero when $n$ is even.
  When $n$ is odd, show that there exist even maps of any given even
  degree. [Hints: If $f$ is even, it factors as a composition
  $S^n\rightarrow\bb{R}P^n\rightarrow S^n$. Using the calculation of
  $H_n(\bb{R}P^n)$ in the text, show that the induced map
  $H_n(S^n)\rightarrow H_n(\bb{R}P^n)$ sends a generator to twice a
  generator when $n$ is odd. It may be helpful to show that the
  quotient map $\bb{R}P^n\rightarrow \bb{R}P^n/\bb{R}P^{n-1}$ induces an
  isomorphism on $H_n$ when $n$ is odd.]
\end{problem}

\begin{proof}
    Let $f$ be a map from $S^n$ to $S^n$ such that $f(x)=f(-x)$ for all $x\in S^n$.
  Since $\bb{R}P^n$ is a quotient space of $S^n$ where antipodal points are
  identified any even map from $S^n$ respects equivalence classes for the
  quotient space and as such it factors as
  \[
    \xymatrix{
      S^n \ar[d]^q \ar[r]^f & S^n\\
      \bb{R}P^n \ar[ru]_{\widetilde{f}}
    }
  \]

  First note that $\deg q =2$. This is because
  for any point $x\in\bb{R}P^n$ it will have two points mapping to it from $S^n$ and
  it will be the identity map giving us local degrees of 1 which add up to 2. Since
  any even map will factor in this way any even map must have even degree.
  
  When $n$ is even $H_n(\bb{R}P^n)\cong\bb{Z}$ and when $n$ is even $H_n(\bb{R}P^n)\cong 0$.
  If $n$ is odd and $f$ is even then $\deg f = \deg q\cdot\deg \widetilde{f}$. However
  since $H_n(\bb{R}P^n)\cong 0$ when $n$ is even then the degree of $f$ has to be zero
  as $\deg \widetilde{f}=0$ since it is mapping out of the trivial group.

  Now suppose that $n$ is odd. There is a quotient map
  $r:\bb{R}P^n\rightarrow (\bb{R}P^n/\bb{R}P^{n-1}\equiv S^n)$. This map will have degree $1$ and
  as such $q\circ r$ will be a map from $S^n\rightarrow S^n$ of degree two. From there if we
  take a map of degree $k$, $f_k:S^n\rightarrow S^n$ (Hatcher $2.32$). Then $f_k\circ q\circ r$ will be
  an even map of degree $2k$.
\end{proof}

\sk

\begin{problem}[20]
  For finite CW complexes $X$ and $Y$, show that
  $\chi(X\times Y)=\chi(X)\chi(Y)$.
\end{problem}

\begin{proof}
  Let $x_n$ and $y_n$ denote the $n$-dimensional simplices of $X$ and $Y$
  respectively. Then
  \[
    \chi(X)\chi(Y)=\left(\sum_{n=0}^\infty(-1)^ix_i\right)\left(\sum_{m=0}^\infty(-1)^n\right)=\sum_{0=i=m+n}^\infty(-1)^{i}\left(\sum_{j=0}^ix_jy_{i-j}\right)
  \]

  However since the number of $i$-simplices in $X\times Y$ is $\sum_{j=0}^ix_jy_{i-j}$
  this demonstrates that
  \[
    \chi(X)\chi(Y)=\chi(X\times Y)
  \]
\end{proof}

\sk

%%%%%%%%%%%%%%%%%%%%%%%%%%%%%%%%%%%%%%%%%%%%%%%%%%%%%%%%%%%%%%%%%%%%%%%%%%%%%
\end{document}
