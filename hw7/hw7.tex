\documentclass[10pt]{article}
\usepackage[utf8]{inputenc}
\usepackage{amscd}
\usepackage{amsmath}
\usepackage{amssymb}
\usepackage{amsthm}
\usepackage{listings}
\usepackage{enumerate}
\usepackage[all,cmtip]{xy}

\textwidth=15cm \textheight=22cm \topmargin=0.5cm \oddsidemargin=0.5cm \evensidemargin=0.5cm

\newcommand{\sk}{\vskip 10mm}
\newcommand{\bb}[1]{\mathbb{#1}}
\newcommand{\rH}{\widetilde{H}}

\theoremstyle{plain}
\newtheorem{problem}{Problem}
\newtheorem{lemma}{Lemma}[problem]

\theoremstyle{remark}
\newtheorem{tpart}{}[problem]
\newtheorem*{ppart}{}

\begin{document}

\begin{problem}[28a]
  Use the Mayer-Vietoris sequence to compute the homology groups of the space
  obtained from a torus $S^1\times S^1$ by attaching a M\"obius band via a
  homeomorphism from the boundary circle of the M\"obius band to the
  circle $S^1\times \{ x_0\}$ in the torus. 
\end{problem}

\begin{proof}
  A M\"obius band is homotopy equivalent to a circle, and we know the homology
  of the torus. Thus we have the long exact sequence
  \[
    \xymatrix{
      \cdots \ar[r] & \rH_n(T^2\cap M) \ar[r] & \rH_n(T^2)\oplus \rH_n(M) \ar[r] & \rH_n(X) \ar[r] & \rH_{n-1}(T^2\cap M) \ar[r] & \cdots
    }
  \]

  The intersection $T^2\cap M$ also has the homology of a circle giving us the exact sequence
  \[
    \xymatrix{
      \cdots \ar[r] & \rH_2(M\cap T^2) \ar[r] & \rH_2(T^2) \ar[r] & \rH_2(X) \ar[r] & \rH_1(M\cap T^2) \ar[r] & \rH_1(M)\oplus\rH_1(T^2) \ar[r] & \\
      &\ar[r] & \rH_1(X) \ar[r] & \rH_1(M\cap T^2) \ar[r] & \cdots
    }
  \]
  which when we substitute in we get
  \[
    \xymatrix{
      0 \ar[r] & \bb{Z} \ar[r]^-{i_*,j_*} & \rH_2(X) \ar[r]^-{\partial_*} & \bb{Z} \ar[r]^-{i_*,j_*} & \bb{Z}\oplus\bb{Z}^2 \ar[r]^-{k_*-l_*} & \rH_1(X) \ar[r] & 0
    }
  \]
  Since the right $i_*,j_*$ is injective the kernel is zero. Thus the image of $\partial_*$
  is 0 making the left $i_*,j_*$ an isomorphism. Similarly the fact that the kernel
  of the right $i_*,j_*$ is zero effectively gives us the short exact sequence
  \[
    \xymatrix{
      0 \ar[r] & \bb{Z} \ar[r]^-{i_*,j_*} & \bb{Z}\oplus\bb{Z}^2 \ar[r]^-{k_*-l_*} & \rH_1(X) \ar[r] & 0
    }
  \]
  which implies that $\rH_1(X)\cong \langle x,y,z\rangle/\langle 2x-y\rangle\cong\langle x,z\rangle\cong\bb{Z}^2$.

  Therefore the homology of $X$ is
  \[
    H_n(X) =
    \left\{
      \begin{array}{cr}
        \bb{Z} & n=0,2\\
        \bb{Z}\oplus\bb{Z} & n=1\\
        0 & \text{otherwise}
      \end{array}
    \right.
  \]
  
\end{proof}

\sk

\begin{problem}[29]
  The surface $M_g$ of genus $g$, embedded in $\bb{R}^3$ in the standard way,
  bounds a compact region $R$. Two copies of $R$, glued together by the identity
  map between boundary surfaces $M_g$, form a closed $3$-manifold $X$.
  Compute the homology groups of $X$ into two copies of $R$. Also compute the
  relative groups $H_i(R,M_g)$.
\end{problem}

\begin{proof}
  
\end{proof}

\sk

\begin{problem}[30]
  For the mapping torus $T_f$ of a map $f: X\rightarrow X$,
  we constructed in Example 2.48 a long exact sequence
  \[
    \xymatrix{
      \cdots \ar[r] & H_n(X) \ar[r]^{\mathrm{id}-f_*} & H_n(X) \ar[r] & H_n(T_f) \ar[r] & H_{n-1}(X) \ar[r] & \cdots
    }
  \]
  Use this to compute the homology of the mapping tori of the following maps:
  \begin{enumerate}
  \item[(a)] A reflection $S^2\rightarrow S^2$
  \item[(b)] A map $S^2\rightarrow S^2$ of degree $2$.
  \item[(c)] The map $S^1\times S^1\rightarrow S^1\times S^1$ that is the
    identity on one factor and a reflection on the other.
  \item[(d)] The map $S^1\times S^1\rightarrow S^1\times S^1$ that is a
    reflection on each factor.
  \item[(e)] The map $S^1\times S^1\rightarrow S^1\times S^1$ that interchanges
    the two factors and then reflects one of the factors.
  \end{enumerate}
\end{problem}

\begin{proof}
  
\end{proof}

\sk

%%%%%%%%%%%%%%%%%%%%%%%%%%%%%%%%%%%%%%%%%%%%%%%%%%%%%%%%%%%%%%%%%%%%%%%%%%%%%
\end{document}
