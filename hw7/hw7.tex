\documentclass[10pt]{article}
\usepackage[utf8]{inputenc}
\usepackage{amscd}
\usepackage{amsmath}
\usepackage{amssymb}
\usepackage{amsthm}
\usepackage{listings}
\usepackage{enumerate}
\usepackage[all,cmtip]{xy}

\textwidth=15cm \textheight=22cm \topmargin=0.5cm \oddsidemargin=0.5cm \evensidemargin=0.5cm

\newcommand{\sk}{\vskip 10mm}
\newcommand{\bb}[1]{\mathbb{#1}}
\newcommand{\rH}{\widetilde{H}}
\newcommand{\id}{\mathrm{id}}

\theoremstyle{plain}
\newtheorem{problem}{Problem}
\newtheorem{lemma}{Lemma}[problem]

\theoremstyle{remark}
\newtheorem{tpart}{}[problem]
\newtheorem*{ppart}{}

\begin{document}

\begin{problem}[28a]
  Use the Mayer-Vietoris sequence to compute the homology groups of the space
  obtained from a torus $S^1\times S^1$ by attaching a M\"obius band via a
  homeomorphism from the boundary circle of the M\"obius band to the
  circle $S^1\times \{ x_0\}$ in the torus. 
\end{problem}

\begin{proof}
  A M\"obius band is homotopy equivalent to a circle, and we know the homology
  of the torus. Thus we have the long exact sequence
  \[
    \xymatrix{
      \cdots \ar[r] & \rH_n(T^2\cap M) \ar[r] & \rH_n(T^2)\oplus \rH_n(M) \ar[r] & \rH_n(X) \ar[r] & \rH_{n-1}(T^2\cap M) \ar[r] & \cdots
    }
  \]

  The intersection $T^2\cap M$ also has the homology of a circle giving us the exact sequence
  \[
    \xymatrix{
      \cdots \ar[r] & \rH_2(M\cap T^2) \ar[r] & \rH_2(T^2) \ar[r] & \rH_2(X) \ar[r] & \rH_1(M\cap T^2) \ar[r] & \rH_1(M)\oplus\rH_1(T^2) \ar[r] & \\
      &\ar[r] & \rH_1(X) \ar[r] & \rH_1(M\cap T^2) \ar[r] & \cdots
    }
  \]
  which when we substitute in we get
  \[
    \xymatrix{
      0 \ar[r] & \bb{Z} \ar[r]^-{i_*,j_*} & \rH_2(X) \ar[r]^-{\partial_*} & \bb{Z} \ar[r]^-{i_*,j_*} & \bb{Z}\oplus\bb{Z}^2 \ar[r]^-{k_*-l_*} & \rH_1(X) \ar[r] & 0
    }
  \]
  Since the right $i_*,j_*$ is injective the kernel is zero. Thus the image of $\partial_*$
  is 0 making the left $i_*,j_*$ an isomorphism. Similarly the fact that the kernel
  of the right $i_*,j_*$ is zero effectively gives us the short exact sequence
  \[
    \xymatrix{
      0 \ar[r] & \bb{Z} \ar[r]^-{i_*,j_*} & \bb{Z}\oplus\bb{Z}^2 \ar[r]^-{k_*-l_*} & \rH_1(X) \ar[r] & 0
    }
  \]
  which implies that $\rH_1(X)\cong \langle x,y,z\rangle/\langle 2x-y\rangle\cong \bb{Z}\oplus\bb{Z}$.

  Therefore the homology of $X$ is
  \[
    H_n(X) =
    \left\{
      \begin{array}{cr}
        \bb{Z} & n=0,2\\
        \bb{Z}\oplus\bb{Z} & n=1\\
        0 & \text{otherwise}
      \end{array}
    \right.
  \]
  
\end{proof}

\sk

\begin{problem}[29]
  The surface $M_g$ of genus $g$, embedded in $\bb{R}^3$ in the standard way,
  bounds a compact region $R$. Two copies of $R$, glued together by the identity
  map between boundary surfaces $M_g$, form a closed $3$-manifold $X$.
  Compute the homology groups of $X$ into two copies of $R$. Also compute the
  relative groups $H_i(R,M_g)$.
\end{problem}

\begin{proof}
  First note that the homology of $M_g$ is
  \[
    H_i(M_g) = 
    \left\{
      \begin{array}{cr}
        \bb{Z} & i=0,2\\
        \bb{Z}^{2g} & i=1\\
        0 & \text{otherwise}
      \end{array}
    \right.
  \]

  Since the region $R$ has the homotopy type of the wedge of $g$ circles
  its homology is
  \[
    H_i(R) = 
    \left\{
      \begin{array}{cr}
        \bb{Z} & i=0\\
        \bb{Z}^g & i=1\\
        0 & \text{otherwise}\\
      \end{array}
    \right.
  \]

  Now we represent $X$ as the union of two slightly thickened copies
  of $R$ such that the intersection is $M_g$. Then using the Mayer-Vietoris
  sequence we get:
  \[
    \xymatrix{
      0 \ar[r] & H_3(X) \ar[r]^-{\partial_*} & H_2(M_g) \ar[r] & 0 \ar[r] & H_2(X) \ar[r]^-{\partial_*} & H_1(M_g) \ar[r]^-{i_*,j_*} & H_1(R)\oplus H_1(R) \ar[r]^-{k_*-l_*} & H_1(X) \ar[r] & 0
    }
  \]

  It immediately follows that $H_3(X)$ is isomorphic to $H_2(M_g)\cong \bb{Z}$. Moreover we know
  have an exact sequence which if we substitute in we get:
  \[
    \xymatrix{
      0 \ar[r] & H_2(X) \ar[r]^-{\partial_*} & \bb{Z}^{2g} \ar[r]^-{i_*,j_*} & \bb{Z}^g \oplus \bb{Z}^g \ar[r]^-{k_*-l_*} & H_1(X) \ar[r] & 0
    }
  \]
  Since $\partial_*$ is injective from its location in the sequence it follows that
  $H_2(X)\cong\ker i_*,j_*$. Looking at the map $i_*,j_*$ it is the inclusion of $M_g$ into both
  copies of $R$. As such half of the meridians will get sent to zero while the others will
  get sent to the generators of $H_1(R)$. Thus the kernel of $i_*,j_*$ is $\bb{Z}^g\cong H_1(X)$.

  Next since $k_*-l_*$ is surjective we have that $H_2(X)\cong(\bb{Z}^g\oplus\bb{Z}^g)/\text{im\ }i_*,j_*$.
  The elements in the image of $i_*,j_*$ are those of the form $(a_1,\ldots,a_g,a_1,\ldots,a_g)$.
  Thus the quotient will be $\bb{Z}^{2g}/\bb{Z}^g\cong\bb{Z}^g\cong H_2(X)$.

  Therefore the homology of $X$ is
  \[
    H_i(X) = 
    \left\{
      \begin{array}{cr}
        \bb{Z} & i=0,3\\
        \bb{Z}^g & i=1,2\\
        0 & \text{otherwise}
      \end{array}
    \right.
  \]

  \sk

  To calculate $H_i(R,M_g)$ we use the long exact sequence from hatcher to get
  \[
    \xymatrix{
      0 \ar[r] & H_3(R,M_g) \ar[r] & H_2(M_g) \ar[r] & 0 \ar[r] & H_2(R,M_g) \ar[r] & H_1(M_g) \ar[r] & H_1(R) \ar[r] & H_1(R,M_g) \ar[r] & 0
    }
  \]

  It follows immediately that $H_3(R,M_g)\cong H_2(M_g)\cong \bb{Z}^{2g}$. In addition we also get
  the exact sequence
  \[
    \xymatrix{
      0 \ar[r] & H_2(R,M_g) \ar[r]^-{\partial_*} & \bb{Z}^{2g} \ar[r]^-{i_*} & \bb{Z}^g \ar[r]^-{q_*} & H_1(R,M_g) \ar[r] & 0
    }
  \]

  The setup is the same as above. Thus $H_2(R,M_g)\cong \ker i_*\cong\bb{Z}^g$. In addition
  $H_1(R,M_g)\cong \bb{Z}^g/\text{im\ }i_*$. However unlike in the previous problem $i_*$
  is surjective. As such $H_1(R,M_g)\cong \bb{Z}^g/\bb{Z}^g\cong 0$. The zeroeth homology is
  zero as $M_g$ touches the single path component of $R$.

  Therefore the relative homology of $R$ with respect to $M_g$ is
  \[
    H_i(R,M_g) = 
    \left\{
      \begin{array}{cr}
        \bb{Z} & i=3\\
        \bb{Z}^g & i=2\\
        0 & \text{otherwise}
      \end{array}
    \right.
  \]
\end{proof}

\sk

\begin{problem}[30]
  For the mapping torus $T_f$ of a map $f: X\rightarrow X$,
  we constructed in Example 2.48 a long exact sequence
  \[
    \xymatrix{
      \cdots \ar[r] & H_n(X) \ar[r]^{\id-f_*} & H_n(X) \ar[r] & H_n(T_f) \ar[r] & H_{n-1}(X) \ar[r] & \cdots
    }
  \]
  Use this to compute the homology of the mapping tori of the following maps:
  \begin{enumerate}
  \item[(a)] A reflection $S^2\rightarrow S^2$
  \item[(b)] A map $S^2\rightarrow S^2$ of degree $2$.
  \item[(c)] The map $S^1\times S^1\rightarrow S^1\times S^1$ that is the
    identity on one factor and a reflection on the other.
  \item[(d)] The map $S^1\times S^1\rightarrow S^1\times S^1$ that is a
    reflection on each factor.
  \item[(e)] The map $S^1\times S^1\rightarrow S^1\times S^1$ that interchanges
    the two factors and then reflects one of the factors.
  \end{enumerate}
\end{problem}

\begin{proof}
  \begin{enumerate}
  \item[(a)] Note that $H_1(T_f)=0$ as it is between $H_1(S^2)$ and $\rH_0(S^2)$.
    The same holds for $n>3$. All of the possible nonzero items are contained
    in the exact sequence
    \[
      \xymatrix{
        0\ar[r] & H_3(T_f) \ar[r] & H_2(S^2) \ar[r] & H_2(S^2) \ar[r] & H_2(T_f) \ar[r] & 0
      }
    \]
    As before this gives us that $H_3(T_f)\cong \ker (\id_*-f_*)$ and
    $H_2(T_f)\cong \bb{Z}/\text{im\ }(\id_*-f_*)$.

    If we evaluate $(\id_*-f_*)(1)$ this is equal to $1-(-1)=2$.
    The kernel of this is trivial and the image is $2\bb{Z}$. Thus the
    homology of $T_f$ is
    \[
      H_i(T_f) =
      \left\{
        \begin{array}{cr}
          \bb{Z} & i=0\\
          \bb{Z}_2 & i=2\\
          0 & \text{otherwise}\\
        \end{array}
      \right.
    \]
  \item[(b)] Same as (a) but with a different map. If we calculate $(\id_*-f_*)(1)$
    this is $1-2=-1$. The kernel of this map is trivial and the image is $\bb{Z}$. Thus
    the homology is:
    \[
      H_i(T_f) =
      \left\{
        \begin{array}{cr}
          \bb{Z} & i=0\\
          0 & \text{otherwise}\\
        \end{array}
      \right.
    \]
  \item[(c)]
    The exact sequence we get from the long exact sequence above is:
    \[
      \xymatrix{
        0 \ar[r] & H_3(T_f) \ar[r]^-{\partial_*} & H_2(T^2)\ar[r]^-{\id_*-f_*} & H_2(T^2) \ar[r]^-{i_*} & H_2(T_f) \ar[r]^-{\partial_*} & H_1(T^2) \ar[r]^-{\id_*-f_*} & H_1(T^2) \ar[r]^-{i_*} & H_1(T_f) \ar[r] & 0
      }
    \]
    If we substitute in for the known groups we get:
    \[
      \xymatrix{
        0 \ar[r] & H_3(T_f) \ar[r]^-{\partial_*} & \bb{Z} \ar[r]^-{\id_*-f_*} & \bb{Z} \ar[r]^-{i_*} & H_2(T_f) \ar[r]^-{\partial_*} & \bb{Z}^2 \ar[r]^-{\id_*-f_*} & \bb{Z}^2 \ar[r]^-{i_*} & H_1(T_f) \ar[r] & 0
      }
    \]

    We can solve for $H_3(T_f)$ and $H_1(T_f)$ in the same manner as above.
    The map induced on $H_2(T^2)$ be $\id_*-f_*$ can be seen from
    $(\id_*-f_*)(1)=1-(-1)=2$. Thus the kernel is trivial making
    $H_3(T_f)=0$. Similarly $H_1(T_f)\cong\bb{Z}^2/\text{im}(id_*-f_*)$. In the
    left case the map applied to the generators gives $(id_*-f_*)(1,0)=(0,0)$
    and $(id_*-f_*)(0,1)=2$. This gives us that $H_1(T_f)\cong\langle x,y\rangle/\langle 2y\rangle\cong \bb{Z}\oplus\bb{Z}_2$.

    Lastly for $H_2(T_f)$ note that since the kernel of the right $\id_*-f_*$ is
    a copy of $\bb{Z}$ we have that $\partial_*$ is surjective onto said copy of $\bb{Z}$.
    This gives us that $\bb{Z}\cong H_2(T_f)/(\ker \partial_*)$. However since the kernel of
    the left $i_*$ is $2\bb{Z}$ (since the image of the left $\id_*-f_*$ is $2\bb{Z}$)
    it must be that $\ker \partial_*\cong\bb{Z}_2$. Which gives us that $H_2(T_f)\cong \bb{Z}\oplus\bb{Z}_2$.

    Therefore the homology of $T_f$ is
    
    \[
      H_i(T_f) =
      \left\{
        \begin{array}{cr}
          \bb{Z} & i=0\\
          \bb{Z}\oplus\bb{Z}_2 & i=1,2\\
          0 & \text{otherwise}\\
        \end{array}
      \right.
    \]
  \item[(d)]
    Once again we start with the exact sequence
    \[
      \xymatrix{
        0 \ar[r] & H_3(T_f) \ar[r]^-{\partial_*} & \bb{Z} \ar[r]^-{\id_*-f_*} & \bb{Z} \ar[r]^-{i_*} & H_2(T_f) \ar[r]^-{\partial_*} & \bb{Z}^2 \ar[r]^-{\id_*-f_*} & \bb{Z}^2 \ar[r]^-{i_*} & H_1(T_f) \ar[r] & 0
      }
    \]

    However this time the left $(id_*-f_*)(1)$ is $1-1=0$. This induces
    an isomorphism $H_3(T_f)\cong H_2(T^2)\cong \bb{Z}$. This effectively shortens our sequence
    to

    \[
      \xymatrix{
        0 \ar[r] & \bb{Z} \ar[r]^-{i_*} & H_2(T_f) \ar[r]^-{\partial_*} & \bb{Z}^2 \ar[r]^-{\id_*-f_*} & \bb{Z}^2 \ar[r]^-{i_*} & H_1(T_f) \ar[r] & 0
      }
    \]

    However as we've seen above, reflection causes $id_*-f_*$ to multiply the cycles of $H_1(T^2)$.
    Thus this map is injective making $\partial_*$ the zero map. This once more induces an
    isomorphism with $i_*$ between $\bb{Z}$ and $H_2(T_f)$.

    Finally we have now shortened our exact sequence to a short
    exact sequence
    
    \[
      \xymatrix{
        0 \ar[r] &\bb{Z}^2 \ar[r]^-{\id_*-f_*} & \bb{Z}^2 \ar[r]^-{i_*} & H_1(T_f) \ar[r] & 0
      }
    \]

    Which gives us that $H_1(T_f)\cong \bb{Z}^2/(2\bb{Z}\oplus 2\bb{Z})\cong\bb{Z}_2\oplus\bb{Z}_2$.

    Therefore the homology of $T_f$ is
    
    \[
      H_i(T_f) =
      \left\{
        \begin{array}{cr}
          \bb{Z} & i=0,2,3\\
          \bb{Z}_2\oplus\bb{Z}_2 & i=1\\
          0 & \text{otherwise}\\
        \end{array}
      \right.
    \]
  \item[(e)]
    As before we have
    \[
      \xymatrix{
        0 \ar[r] & H_3(T_f) \ar[r]^-{\partial_*} & \bb{Z} \ar[r]^-{\id_*-f_*} & \bb{Z} \ar[r]^-{i_*} & H_2(T_f) \ar[r]^-{\partial_*} & \bb{Z}^2 \ar[r]^-{\id_*-f_*} & \bb{Z}^2 \ar[r]^-{i_*} & H_1(T_f) \ar[r] & 0
      }
    \]

    This will also cause the left $\id_*-f_*$ to be the zero map giving us
    $H_3(T_f)\cong\bb{Z}$. Once again shortening our sequence to

    \[
      \xymatrix{
        0 \ar[r] & \bb{Z} \ar[r]^-{i_*} & H_2(T_f) \ar[r]^-{\partial_*} & \bb{Z}^2 \ar[r]^-{\id_*-f_*} & \bb{Z}^2 \ar[r]^-{i_*} & H_1(T_f) \ar[r] & 0
      }
    \]

    Note that the right instance of $\id_*-f_*$ is injective
    as $(\id_*-f_*)(1,0)=(0,1)$ and $(\id_*-f_*)(0,1)=(-1,0)$. Just as
    before this induces an isomorphism on $H_2(T_f)\cong \bb{Z}$ and giving
    us a short exact sequence

    \[
      \xymatrix{
        0 \ar[r] &\bb{Z}^2 \ar[r]^-{\id_*-f_*} & \bb{Z}^2 \ar[r]^-{i_*} & H_1(T_f) \ar[r] & 0
      }
    \]

    However from above we can see that $\id_*-f_*$ is an isomorphism which
    forces $H_1(T_f)$ to be zero. Therefore the homology of $T_f$ is
    
    \[
      H_i(T_f) =
      \left\{
        \begin{array}{cr}
          \bb{Z} & i=0,2,3\\
          0 & \text{otherwise}\\
        \end{array}
      \right.
    \]
  \end{enumerate}
\end{proof}

\sk

%%%%%%%%%%%%%%%%%%%%%%%%%%%%%%%%%%%%%%%%%%%%%%%%%%%%%%%%%%%%%%%%%%%%%%%%%%%%%
\end{document}
