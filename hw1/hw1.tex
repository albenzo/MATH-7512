\documentclass[10pt]{article}
\usepackage[utf8]{inputenc}
\usepackage{amscd}
\usepackage{amsmath}
\usepackage{amssymb}
\usepackage{amsthm}
\usepackage{listings}
\usepackage{enumerate}

\textwidth=15cm \textheight=22cm \topmargin=0.5cm \oddsidemargin=0.5cm \evensidemargin=0.5cm

\newcommand{\sk}{\vskip 10mm}
\newcommand{\bb}[1]{\mathbb{#1}}
\newcommand{\ra}{\rightarrow}
\newcommand{\ima}{\mathrm{im}\ }

\theoremstyle{plain}
\newtheorem{problem}{Problem}
\newtheorem{lemma}{Lemma}[problem]

\theoremstyle{remark}
\newtheorem{tpart}{}[problem]
\newtheorem*{ppart}{}

\begin{document}

\begin{problem}[1]
  What familiar space is the quotient $\Delta$-complex of a $2$-simplex
  $[v_0,v_1,v_2]$ obtained by identifying the edges $[v_0,v_1]$ and
  $[v_1,v_2]$, preserving the ordering of vertices?
\end{problem}

Use the following drawing of $X$ as our guide for the boundary maps.

\textbf{Insert pretty picture here}

Since we are identifying edges $e_1$ and $e_2$ this also induces the
identification of $v_1,v_2,$ and $v_3$. We'll call them $e$
and $v$ respectively. Then:
\begin{align*}
  \partial f &= 2e+e_3\\
  \partial e &= v-v = 0\\
  \partial e_3 &= v-v = 0\\
\end{align*}

Then for $H_2(X)$ we have $\ima \partial_3=0$ and $\ker \partial_2 = 0$.
Thus $H_2(X)\cong 0$.

For $H_1$ we have $\ima \partial_2=\langle 2e+e_3\rangle$ and $\ker\partial_1=\langle e,e_3\rangle$.
Then we have
\[ H_1(X)\cong\langle e,e_3\rangle/\langle 2e+e_3\rangle\cong\langle e,e_3|2e=-e_3\rangle\cong\langle e\rangle\cong \bb{Z}\]

Finally $H_0(X)=\bb{Z}$ as there is only a single connected component.

Therefore the homology for $X$ is
\[
  H_p(X)=
  \left\{
    \begin{array}{ll}
      \bb{Z} & p=0,1\\
      0 & p\geq 2\\
    \end{array}
  \right.
\]

It's probably a M\"obius strip.

\sk

\begin{problem}[4]
  Compute the simplicial homology groups of the triangular parachute
  obtained from $\Delta^2$ by identifying its three vertices to a single point.
\end{problem}

Use the picture below to guide the boundary map for $X$.

\textbf{Insert pretty picture here}

The boundary map for our components will be:
\begin{align*}
  \partial f &= e_1+e_2+e_2\\
  \partial e_1 &= v-v=0\\
  \partial e_2 &= v-v=0\\
  \partial e_3 &= v-v=0\\
\end{align*}

For $H_2(X)$ we have $\ima \partial_3=0$ and $\ker\partial_2=0$. Thus
$H_2(X)\cong 0$.

For $H_1(X)$ we have $\ima \partial_2=\langle e_1+e_2+e_3\rangle$ and $\ker\partial_1=\langle e_1,e_2,e_3\rangle$.
Thus
\[ H_1(X)=\langle e_1,e_2,e_3\rangle/\langle e_1+e_2+e_3\rangle\cong\langle e_1,e_2,e_3|e_1+e_2=-e_3\rangle\cong\langle e_1+e_2\rangle\cong\bb{Z}\oplus\bb{Z}\]

Finally there is only a single connected component so $H_0(X)\cong\bb{Z}$.

Therefore the homology for $X$ is

\[
  H_p(X)=
  \left\{
    \begin{array}{ll}
      \bb{Z} & p=0\\
      \bb{Z}\oplus\bb{Z} & p=1\\
      0 & p\geq 2\\
    \end{array}
  \right.
\]

\sk

\begin{problem}[6]
  Compute the simplicial homology groups of the $\Delta$-complex obtained
  from $n+1$ $2$-simplicies $\Delta^2_0, \cdots, \Delta^2_n$ by identifying
  all three edges of $\Delta^2_0$ to a single edge, and for $i>0$ identifying
  the edges $[v_0,v_1]$ and $[v_1,v_2]$ of $\Delta^2_i$ to a single edge and
  the edge of $[v_0,v_2]$ to the edge $[v_0,v_1]$ of $\Delta^2_{i-1}$.
\end{problem}

We'll use the following diagram for the space $X$ to define the homology.

\textbf{Insert a picture here}.

The values for the boundary maps are
\begin{align*}
  \partial e_i &= v_0-v_0=0\\
  \partial f_0 &= 3e_0\\
  \partial f_i &= 2e_i+e_{i-1} & 1\leq i\leq n
\end{align*}

For $H_2(X)$ we have that $\ima \partial_3=0$. For $\ker \partial_2$ start with the matrix
\[
  \left(
    \begin{array}{ccccc}
      3&1&0&\cdots&0\\
      0&2&1&\cdots&\vdots\\
      \vdots&0&\ddots&\ddots&0\\
       &\vdots&&2&1\\
      0&0&\cdots&0&2\\
    \end{array}
  \right)
\]

Then the smith normal form of the matrix is
\[
  \left(
    \begin{array}{ccccc}
      1&0&\cdots&&0\\
      0&\ddots&&&\vdots\\
      \vdots&&&1&0\\
      0&\cdots&&0&3\cdot 2^{n-1}\\
    \end{array}
  \right)
\]

The null space for this matrix is trivial. Thus $\ker \partial_2=0$ and
thus $H_2(X)\cong 0$.

Then for $H_1(X)$ we have that $\ker\partial_1=\langle e_i\rangle$ and from the above
matrix we know that $\ima \partial_2=\langle e_1,\ldots,e_{n-1},3\cdot 2^{n-1}e_n\rangle$
giving us that $H_1(X)\cong \bb{Z}_{3\cdot 2^{n-1}}$.

Finally $H_0(X)\cong \bb{Z}$ as there is only one connected component.

Therefore the homology of $X$ is

\[
  H_p(X) =
  \left\{
    \begin{array}{lr}
      \bb{Z}& p=0\\
      \bb{Z}_{3\cdot 2^{n-1}} & p=1\\
      0 & p\geq 2\\
    \end{array}
  \right.
\]


%%%%%%%%%%%%%%%%%%%%%%%%%%%%%%%%%%%%%%%%%%%%%%%%%%%%%%%%%%%%%%%%%%%%%%%%%%%%% 
\end{document}
