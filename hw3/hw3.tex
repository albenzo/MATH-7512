\documentclass[10pt]{article}
\usepackage[utf8]{inputenc}
\usepackage{amscd}
\usepackage{amsmath}
\usepackage{amssymb}
\usepackage{amsthm}
\usepackage{listings}
\usepackage{enumerate}
\usepackage[all,cmtip]{xy}

\textwidth=15cm \textheight=22cm \topmargin=0.5cm \oddsidemargin=0.5cm \evensidemargin=0.5cm

\newcommand{\sk}{\vskip 10mm}
\newcommand{\bb}[1]{\mathbb{#1}}
\newcommand{\ra}{\rightarrow}
\newcommand{\ima}{\mathrm{im}\ }

\theoremstyle{plain}
\newtheorem{problem}{Problem}
\newtheorem{lemma}{Lemma}[problem]

\theoremstyle{remark}
\newtheorem{tpart}{}[problem]
\newtheorem*{ppart}{}

\begin{document}

\begin{problem}[15]
  For an exact sequence $A\rightarrow B\rightarrow C\rightarrow D\rightarrow E$ show that $C=0$ iff
  the map $A\rightarrow B$ is surjective and $D\rightarrow E$ is injective. Hence
  for a pair of spaces $(X,A)$, the inclusion induces isomorphisms
  on all homology groups iff $H_n(X,A)=0$ for all $n$.
\end{problem}

\begin{proof}
  Label the maps of the exact sequence as:
  \[
    \xymatrix{A \ar[r]^\alpha & B \ar[r]^\beta & C \ar[r]^\gamma & D \ar[r]^\delta & E}
  \]
  
  Suppose that $C=0$. Then $\ker \beta= B$ which by exactness gives
  $\ima \alpha= B$ implying that $\alpha$ is surjective. On the other hand
  $\ima \gamma=0$ which by exactness gives $\ker \delta=0$ and as such
  $\delta$ is injective.

  Now suppose that $\alpha$ is surjective and that $\delta$ is injective. Then
  $\ima \alpha = B$ which implies that $\ker \beta=B$. In addition, since
  $\ker\delta=0$ by exactness we have that $\ima \gamma=0$. However since
  the image of $\beta$ is $0$ by exactness $\ker \gamma=0$. Since the
  kernel of $\gamma$ is 0 and the image is 0 it must be that the group
  $C$ is zero.

  Therefore $C=0$ if, and only if, $\alpha$ is surjective and $\delta$ is
  injective.
\end{proof}

\sk

\begin{problem}[16]
  \begin{enumerate}
  \item[(a)] Show that $H_0(X,A)=0$ iff $A$ meets each path-component
    of $X$.
  \item[(b)] Show that $H_1(X,A)=0$ iff $H_1(A)\rightarrow H_1(X)$ is surjective and
    each path component contains at most one path-component of $A$.
  \end{enumerate}
\end{problem}

\begin{proof}\ \\
  \begin{enumerate}
  \item[(a)] Note that $H_0(X,A)=C_0(X,A)/\ima\partial_1$. It then follows that
    $H_0(X,A)=0$ only when $C_0(X,A)=0$ or $\ima\partial_1=C_0(X,A)$. If $C_0(X,A)=0$
    then $A=X$ and the problem follows immediately. Otherwise it must
    be that $C_0(X,A)\neq 0$.

    Now suppose that $A$ meets each path component of $X$. Then each
    point $x\in X$ will have a path connecting it to $A$ which implies that
    $x\in\ima \partial_1$. Since this holds for any point we have that $\ima\partial_1=C_0(X,A)$
    and as such $H_0(X,A)=0$.

    Otherwise if $H_0(X,A)=0$ then $\ima\partial_1=C_0(X,A)$ meaning that any
    point $x\in X$ has a one-simplex that ends at $x$ and inside of $A$.
    This creates a path from $x$ to $A$ which implies that $A$ must
    contact each path component of $X$.
  \item[(b)] From Hatcher we have the long exact sequence
    \[
      \xymatrix{\cdots\ar[r]&H_1(A)\ar[r]^{i_*} & H_1(X) \ar[r]^{j_*} & H_1(X,A) \ar[r]^{\partial_*} &
        H_0(A) \ar[r]^{i_*} & H_0(X) \ar[r]^{j_*} & H_0(X,A) \ar[r] & 0}
    \]

    First suppose that $i_*:H_1(A)\rightarrow H_1(X)$ is surjective. Then by exactness
    $\ker j_*=H_1(X)$ and it then follows that $\ima j_*=0$ which means
    that $\ker\partial_*=0$. This implies that $\partial_*$ is injective. If the kernel
    of $i_*:H_0(A)\rightarrow H_0(X)$ is trivial this will imply that $H_1(X,A)=0$ as
    $\ima \partial=0$. However since $A$ touches each path component of $X$ at
    most once the induced map $i_{*1}$ will be injective which gives it a
    trivial kernel. Thus if $i_*:H_1(A)\rightarrow H_1(X)$ is surjective and $A$ meets
    each path component of $X$ at most once we have that $H_1(X,A)=0$.

    Otherwise suppose that $H_1(X,A)=0$. Then by exactness we have
    that $i_{*1}$ is injective and as such $A$ will meet each
    path component of $X$ at most once. Also $\ker j_*=H_1(X)$ this
    implies that $\ima i_{*2}=H_1(X)$ making it surjective.

    Therefore $H_1(X,A)=0$ iff $H_1(A)\rightarrow H_1(X)$ is surjective and each path
    component contains at most one path-component of $A$.
  \end{enumerate}
\end{proof}

\sk

\begin{problem}[18]
  Show that for the subspace $\bb{Q}\subset\bb{R}$, the relative homology group
  $H_1(\bb{R},\bb{Q})$ is free abelian and find a basis.
\end{problem}

\begin{proof}
  From Hatcher we have the long exact sequence
  \[
    \xymatrix{\cdots \ar[r] & H_1(\bb{R})\ar[r] & H_1(\bb{R},\bb{Q})\ar[r]^{\partial_*}
      & H_0(\bb{Q})\ar[r]^{i_*} & H_0(\bb{R}) \ar[r] & H_)(\bb{R},\bb{Q}) \ar[r] & 0 }
  \]

  We know the homology of $\bb{R}$ since it is contractible and we know that
  $H_0(\bb{R},\bb{Q})$ is trivial by the previous problem. Moreover since
  $\bb{Q}$ is completely disconnected its zeroth homology will be a copy
  of $\bb{Z}$ for each rational number. This gives us the short exact sequence:
  \[
    \xymatrix{0 \ar[r] & H_1(\bb{R},\bb{Q})\ar[r]^{\partial_*}&\bigoplus_{q\in\bb{Q}}\bb{Z}\ar[r]^-{i_*}
      &\bb{Z}\ar[r] & 0 }
  \]

  We can write elements of $\bigoplus_{q\in\bb{Q}}\bb{Z}$ as formal sums of rational
  numbers with integer coefficients ($\sum_{q\in\bb{Q}}a_qq$). Then the
  map induced by the inclusion would be
  \[
    i_*\left(\sum a_qq\right) = \sum a_q
  \]

  The kernel of this map is $\{\sum a_qq | \sum a_q=0\}\cong \ima \partial_*$. However since
  $\partial_*$ is injective by the first isomorphism theorem we have that
  \[
    H_1(\bb{R},\bb{Q})\cong \{\sum a_qq|\sum a_q=0\}
  \]

  This group is a subgroup of a free abelian group which implies that it is
  also free abelian. We can write a basis as
  \footnote{Almost did this with $-1\cdot 0+1\cdot q$ but I disliked the notation} 
  \[
    \langle -1\cdot p+1\cdot q| q\in\bb{Q}\setminus\{p\}\rangle
  \]
  
\end{proof}

%%%%%%%%%%%%%%%%%%%%%%%%%%%%%%%%%%%%%%%%%%%%%%%%%%%%%%%%%%%%%%%%%%%%%%%%%%%%%
\end{document}
