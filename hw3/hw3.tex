\documentclass[10pt]{article}
\usepackage[utf8]{inputenc}
\usepackage{amscd}
\usepackage{amsmath}
\usepackage{amssymb}
\usepackage{amsthm}
\usepackage{listings}
\usepackage{enumerate}
\usepackage[all,cmtip]{xy}

\textwidth=15cm \textheight=22cm \topmargin=0.5cm \oddsidemargin=0.5cm \evensidemargin=0.5cm

\newcommand{\sk}{\vskip 10mm}
\newcommand{\bb}[1]{\mathbb{#1}}
\newcommand{\ra}{\rightarrow}
\newcommand{\ima}{\mathrm{im}\ }

\theoremstyle{plain}
\newtheorem{problem}{Problem}
\newtheorem{lemma}{Lemma}[problem]

\theoremstyle{remark}
\newtheorem{tpart}{}[problem]
\newtheorem*{ppart}{}

\begin{document}

\begin{problem}[15]
  For an exact sequence $A\rightarrow B\rightarrow C\rightarrow D\rightarrow E$ show that $C=0$ iff
  the map $A\rightarrow B$ is surjective and $D\rightarrow E$ is injective. Hence
  for a pair of spaces $(X,A)$, the inclusion induces isomorphisms
  on all homology groups iff $H_n(X,A)=0$ for all $n$.
\end{problem}

\begin{proof}
  Label the maps of the exact sequence as:
  \[
    \xymatrix{A \ar[r]^\alpha & B \ar[r]^\beta & C \ar[r]^\gamma & D \ar[r]^\delta & E}
  \]
  
  Suppose that $C=0$. Then $\ker \beta= B$ which by exactness gives
  $\ima \alpha= B$ implying that $\alpha$ is surjective. On the other hand
  $\ima \gamma=0$ which by exactness gives $\ker \delta=0$ and as such
  $\delta$ is injective.

  Now suppose that $\alpha$ is surjective and that $\delta$ is injective. Then
  $\ima \alpha = B$ which implies that $\ker \beta=B$. In addition, since
  $\ker\delta=0$ by exactness we have that $\ima \gamma=0$. However since
  the image of $\beta$ is $0$ by exactness $\ker \gamma=0$. Since the
  kernel of $\gamma$ is 0 and the image is 0 it must be that the group
  $C$ is zero.

  Therefore $C=0$ if, and only if, $\alpha$ is surjective and $\delta$ is
  injective.
\end{proof}

\sk

\begin{problem}[16]
  \begin{enumerate}
  \item[(a)] Show that $H_0(X,A)=0$ iff $A$ meets each path-component
    of $X$.
  \item[(b)] Show that $H_1(X,A)=0$ iff $H_1(A)\rightarrow H_1(X)$ is surjective and
    each path component contains at most one path-component of $A$.
  \end{enumerate}
\end{problem}

\begin{proof}\ \\
  \begin{enumerate}
  \item[(a)] 
  \item[(b)]
  \end{enumerate}
\end{proof}

\sk

\begin{problem}[18]
  Show that for the subspace $\bb{Q}\subset\bb{R}$, the relative homology group
  $H_1(\bb{R},\bb{Q})$ is free abelian and find a basis.
\end{problem}

\begin{proof}

\end{proof}

%%%%%%%%%%%%%%%%%%%%%%%%%%%%%%%%%%%%%%%%%%%%%%%%%%%%%%%%%%%%%%%%%%%%%%%%%%%%%
\end{document}
