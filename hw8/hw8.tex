\documentclass[10pt]{article}
\usepackage[utf8]{inputenc}
\usepackage{amscd}
\usepackage{amsmath}
\usepackage{amssymb}
\usepackage{amsthm}
\usepackage{listings}
\usepackage{enumerate}
\usepackage[all,cmtip]{xy}

\textwidth=15cm \textheight=22cm \topmargin=0.5cm \oddsidemargin=0.5cm \evensidemargin=0.5cm

\newcommand{\sk}{\vskip 10mm}
\newcommand{\bb}[1]{\mathbb{#1}}
\newcommand{\ra}{\rightarrow}

\theoremstyle{plain}
\newtheorem{problem}{Problem}
\newtheorem{lemma}{Lemma}[problem]

\theoremstyle{remark}
\newtheorem{tpart}{}[problem]
\newtheorem*{ppart}{}

\begin{document}

\begin{problem}[2.2.35]
  Use the Mayer-Vietoris sequence to show that a
  nonorientable closed surface, or more generally
  a finite simplicial complex $X$ for which $H_1(X)$
  contains torsion, cannot be embedded as a subspace
  of $\bb{R}^3$ in such a way as to have a neighborhood
  homeomorphic to the mapping cylinder of some map from a
  closed orientable surface to $X$. [This assumption on a
  neighborhood is in fact not needed if one deduces the
  result from Alexander duality in 3.3]
\end{problem}

\begin{proof}
  
\end{proof}

\sk

\begin{problem}[2.2.38]
  Show that the commutative diagram
  \[
    \xymatrix{
      \cdots \ar[r] & C_{n+1} \ar[dd] \ar[rd] & & B_n \ar[dd] \ar[r] & C_n \ar[dd] \ar[rd] & & B_{n-1} \ar[r] &\cdots\\
      && A_n \ar[ru] \ar[rd] & && A_{n-1} \ar[ru] \ar[rd] & \\
      \cdots \ar[r] & E_{n+1} \ar[ru] & & D_n \ar[r] & E_n \ar[ru] & & D_{n-1} \ar[r] &\cdots\\
    }
  \]
  with the two sequences across the top and bottom exact,
  gives rise to an exact sequence
  \[
    \cdots\rightarrow E_{n+1}\rightarrow B_n\rightarrow C_n\oplus D_n\rightarrow E_n\rightarrow B_{n-1}\rightarrow\cdots
  \]
  where the maps are obtained from those in the previous
  diagram in the obvious way, except that $B_n\rightarrow C_n\oplus D_n$
  has a minus sign in one coordinate.
\end{problem}

\begin{proof}
  
\end{proof}

\sk

\begin{problem}[2.B.1]
  Compute $H_i(S^n\setminus X)$ when $X$ is he subspace of $S^n$ homeomorphic to
  $S^k\vee S^\ell$ or to $S^k\coprod S^\ell$.
\end{problem}

\begin{proof}
  
\end{proof}

\sk

%%%%%%%%%%%%%%%%%%%%%%%%%%%%%%%%%%%%%%%%%%%%%%%%%%%%%%%%%%%%%%%%%%%%%%%%%%%%%
\end{document}
