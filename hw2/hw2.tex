\documentclass[10pt]{article}
\usepackage[utf8]{inputenc}
\usepackage{amscd}
\usepackage{amsmath}
\usepackage{amssymb}
\usepackage{amsthm}
\usepackage{listings}
\usepackage{enumerate}
\usepackage{graphicx}
\usepackage[all,cmtip]{xy}

\graphicspath{{images/}}

\textwidth=15cm \textheight=22cm \topmargin=0.5cm \oddsidemargin=0.5cm \evensidemargin=0.5cm

\newcommand{\sk}{\vskip 10mm}
\newcommand{\bb}[1]{\mathbb{#1}}
\newcommand{\ra}{\rightarrow}
\newcommand{\ima}{\mathrm{im}\ }

\theoremstyle{plain}
\newtheorem{problem}{Problem}
\newtheorem{lemma}{Lemma}[problem]

\theoremstyle{remark}
\newtheorem{tpart}{}[problem]
\newtheorem*{ppart}{}

\begin{document}

\begin{problem}[11]
  
\end{problem}

\begin{proof}
  Let $A$ be a retract of $X$. Then there is a map such that
  $r:X\rightarrow A$ which when composed with inclusion we have
  $r\circ i= id_A$. However when we look at the induced maps on
  homology we see that $r_*\circ i_*=id_{A*}$. As such $i_*$ has
  a left inverse and it then follows that $i_*$ is injective.
\end{proof}

\sk

\begin{problem}[12]
  Show that chain homotopy of chain maps is an equivalence relation.
\end{problem}

\begin{proof}
  Let $f_\#\sim g_\#$ denote that $f_\#,g_\#$ are chain homotopic. Now we verify
  that chain homotopy is an equivalence relation.

  \begin{itemize}
  \item
    Let $P$ be the trivial map. Then $\partial P + P\partial=0=f_\#-f_\#$. Thus
    $\sim$ is reflexive.
  \item
    Suppose that $f_\#\sim g_\#$ with prism $P$. Then for $-P$ we
    have that
    \[(\partial -P)+(-P)\partial = -(\partial P+P\partial)=-(f_\#-g_\#)=g_\#-f_\#\]
    Thus $\sim$ is reflexive.
  \item
    Finally let $f_\#\sim g_\#$ and $g_\#\sim h_\#$ with prism $P$ and $Q$ respectively.
    Then
    \[ \partial(P+Q) +(P+Q)\partial = \partial P + P\partial+\partial Q + Q\partial=f_\#-g_\#+g_\#-h_\#=f_\#-h_\# \]
    Thus $\sim$ is transitive.
  \end{itemize}

  Therefore, since it is reflexive, symmetric, and transitive it is an equivalence relation.
\end{proof}

\sk

\begin{problem}[14]
  
\end{problem}

For this problem we'll be implicitly assuming the structure theorem of finite
generated abelian groups to eliminate choices.


In order for the short exact sequence
\[
  \xymatrix{0 \ar[r] & \bb{Z}_4 \ar[r] & \bb{Z}_8\oplus \bb{Z}_2\ar[r] &\bb{Z}_4 \ar[r] & 0}
\]
to exist we need for $\bb{Z}_4$ to be expressible as a quotient of
$\bb{Z}_8\oplus \bb{Z}_2$ and $\bb{Z}_4$. However this cannot occur as there is only
a single component to $\bb{Z}_4$ the homomorphism needs to be sent to either
the $\bb{Z}_8$ or the $\bb{Z}_2$. If it is sent to the $\bb{Z}_8$ we can get it
down to $\bb{Z}_2$ but $\bb{Z}_2\oplus\bb{Z}_2\ncong \bb{Z}_4$. Similarly if
$\bb{Z}_4$ is sent to $\bb{Z}_2$ we can eliminate that component but
$\bb{Z}_8\ncong \bb{Z}_4$. Therefore the above short exact sequence cannot
exist.

For the short exact sequence
\[
  \xymatrix{0 \ar[r] & \bb{Z}_{p^m} \ar[r] & A \ar[r] & \bb{Z}_{p^n}\ar[r] & 0}
\]
to exist we need for $\bb{Z}_{p^n}\cong A/\bb{Z}_{p^m}$. Since $p$ is prime the
only possibilities for this are $A\cong \bb{Z}_{p^m}\oplus\bb{Z}_{p^n}$ and sending
$\bb{Z}_{p^m}$ to itself or $\bb{Z}_{p^{m+n}}$ and quotienting out the $\bb{Z}_{p^m}$.


Finally for the short exact sequence
\[
  \xymatrix{0 \ar[r] & \bb{Z} \ar[r] & B \ar[r] & \bb{Z}_n\ar[r] & 0}
\]

to exist, as above, we need $\bb{Z}_n\cong B/\bb{Z}$. Since the first map
must be injective due to its position we need to send $\bb{Z}$ into
some copy of $\bb{Z}$. Thus the possibilities are $B\cong \bb{Z}\oplus\bb{Z}_n$
quotienting out the $\bb{Z}$ or $B\cong\bb{Z}$ and sending $\bb{Z}$ to $n\bb{Z}$.

\sk

%%%%%%%%%%%%%%%%%%%%%%%%%%%%%%%%%%%%%%%%%%%%%%%%%%%%%%%%%%%%%%%%%%%%%%%%%%%%% 
\end{document}
