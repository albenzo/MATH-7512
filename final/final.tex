\documentclass[10pt]{article}
\usepackage[utf8]{inputenc}
\usepackage{amscd}
\usepackage{amsmath}
\usepackage{amssymb}
\usepackage{amsthm}
\usepackage{listings}
\usepackage{enumerate}
\usepackage{mathtools}
\usepackage[all,cmtip]{xy}

\textwidth=15cm \textheight=22cm \topmargin=0.5cm \oddsidemargin=0.5cm \evensidemargin=0.5cm

\newcommand{\sk}{\vskip 10mm}
\newcommand{\bb}[1]{\mathbb{#1}}
\newcommand{\ra}{\rightarrow}
\newcommand{\rH}{\widetilde{H}}

\theoremstyle{plain}
\newtheorem{problem}{Problem}
\newtheorem{lemma}{Lemma}[problem]

\theoremstyle{remark}
\newtheorem{tpart}{}[problem]
\newtheorem*{ppart}{}

\begin{document}

\begin{problem}[2.1.17]
  \begin{enumerate}
  \item[(a)] Compute the homology groups $H_n(X,A)$ when $X$ is $S^2$ or
    $S^1\times S^1$ and $A$ is a finite set of points in $X$.
  \item[(b)] Compute the groups $H_n(X,A)$ and $H_n(X,B)$ for $X$ a
    closed orientable surface of genus two with $A$ and $B$ the circles
    shown ($A$ is circle around join of Tori and $B$ is around tube of right
    tori). [What are $X/A$ and $X/B$.]
  \end{enumerate}
\end{problem}

\begin{proof}
  \begin{enumerate}
  \item[(a)] Beginning with $S^2$ use the long exact sequence for homology
    to get
    \[
      0 \rightarrow H_2(S^2)\cong \bb{Z} \rightarrow H_2(S^2,A) \rightarrow 0 \rightarrow H_1(S^2)\cong 0 \rightarrow H_1(S^2,A) \rightarrow H_0(A)\cong\bb{Z}^{|A|} \rightarrow H_0(S^2)\cong\bb{Z} \rightarrow 0
    \]
    From this we can deduce that
    \[
      H_i(S^2,A)\cong \left\{
        \begin{array}{lr}
          \bb{Z}& i=2\\
          \bb{Z}^{|A|-1} & i=1\\
          0 & \text{else}
        \end{array}
      \right.
    \]
    Similarly for $T^2$ we have the long exact sequence
    \[
      0 \rightarrow H_2(T^2)\cong \bb{Z} \rightarrow H_2(T^2,A) \rightarrow 0 \rightarrow H_1(T^2)\cong \bb{Z}^2 \rightarrow H_1(T^2,A) \rightarrow H_0(A)\cong\bb{Z}^{|A|} \rightarrow H_0(T^2)\cong\bb{Z} \rightarrow 0
    \]
    Which gives us the relative homology
    \[
      H_i(T^2,A) = \left\{
        \begin{array}{lr}
          \bb{Z}&i=2\\
          \bb{Z}^{|A|+2}& i=1\\
          0 & \text{else}
        \end{array}
      \right.
    \]
  \item[(b)] Starting with $A$ and after substituting for known groups we have the long exact sequence
    \[
      0 \rightarrow \bb{Z} \rightarrow H_2(\Sigma_2,A)\rightarrow \bb{Z}\rightarrow\bb{Z}^4\rightarrow H_1(\Sigma_2,A)\rightarrow \bb{Z} \rightarrow \bb{Z} \rightarrow 0
    \]
    However the last map $\bb{Z}\rightarrow \bb{Z}$ is an isomorphism since it is the inclusion
    of one connected component to another. So we can shorten our exact sequence to
    \[
      0 \rightarrow \bb{Z} \rightarrow H_2(\Sigma_2,A)\rightarrow \bb{Z}\rightarrow\bb{Z}^4\rightarrow H_1(\Sigma_2,A)\rightarrow 0
    \]
    Finally note that the map from $\bb{Z}$ to $\bb{Z}^4$ is going to be the zero map.
    As such this will force two shorter exact sequences
    \[
      0\rightarrow \bb{Z} \rightarrow H_2(\Sigma_2,A)\rightarrow \bb{Z}\rightarrow 0 , 0\rightarrow \bb{Z}^4\rightarrow H_1(\Sigma_2,A)\rightarrow 0
    \]
    which give us
    \[
      H_i(\Sigma_2,A) = \left\{
        \begin{array}{lr}
          \bb{Z}^2& i=2\\
          \bb{Z}^4 & i=1\\
          0 & \text{else}\\
        \end{array}
      \right.
    \]

    The case for $B$ is similar. We get down to the same exact sequence
    \[
      0 \rightarrow \bb{Z} \rightarrow H_2(\Sigma_2,B)\rightarrow \bb{Z}\rightarrow\bb{Z}^4\rightarrow H_1(\Sigma_2,B)\rightarrow 0
    \]
    However in this case the map $\bb{Z}\rightarrow\bb{Z}^4$ sends $1$ to
    one of the four generators. Since this is injective the map will induce
    an isomorphism for the second homology and will induce the short exact sequence
    \[
      0 \rightarrow \bb{Z} \rightarrow \bb{Z}^4\rightarrow H_1(\Sigma_2,B) \rightarrow 0
    \]
    giving us the homology
    \[
      H_i(\Sigma_2,B) = \left\{
        \begin{array}{lr}
          \bb{Z}& i=2\\
          \bb{Z}^3 &i=1\\
          0 \text{else}\\
        \end{array}
        \right.
    \]
  \end{enumerate}
\end{proof}

\sk

\begin{problem}[2.2.40]
  From the long exact sequence of homology groups associated to the short
  exact sequence of chain complexes
  $0\rightarrow C_i(X)\xrightarrow{n} C_i(X)\xrightarrow{\varphi} C_i(X;\bb{Z}_n)\rightarrow 0$
  deduce immediately that there are short exact sequences
  \[
    0\rightarrow H_i(X)/nH_i(X)\rightarrow H_i(X;\bb{Z}_n)\rightarrow n-Torsion(H_{i-1}(X))\rightarrow 0
  \]
  where $n-Torsion(G)$ is the kernel of the map $G\xrightarrow{n} G$,
  $g\mapsto ng$. Use this to show that $\rH_i(X;\bb{Z}_p)=0$ for all $i$ and all
  primes $p$ iff $\rH_i(X)$ is a vector space over $\bb{Q}$ for all $i$.
\end{problem}

\begin{proof}
  Since we have a short exact sequence of chain complexes we can form
  a long exact sequence in homology of the form
  \[
    \xymatrix{
      \cdots \ar[r]^{\partial_*} & H_i(X) \ar[r]^{n_*} & H_i(X) \ar[r]^{\varphi_*} & H_i(X;\bb{Z}_n) \ar[r] & \cdots
    }
  \]
  From this we can make a short exact sequence
  \[
    0\rightarrow H_i(X)/nH_i(X)\rightarrow H_i(X;\bb{Z}_n)\rightarrow n-Torsion(H_{i-1}(X))\rightarrow 0
  \]
  which will be exact since the first map is quotienting out by the kernel of $\varphi_*$ and
  the latter map is surjective since $n-Torsion$ is defined as the kernel of $n_*$.

  Suppose that $\rH_i(X;\bb{Z}_p)=0$ for all primes $p$. Then $p-Torsion(\rH_i(X))=0$ which
  implies that $\rH_i(X)$ has no torsion. So we can then define an action of $\bb{Q}$ on
  $H_i(X)$ by $(\frac{p}{q})\cdot g = h$ where $h$ is the unique solution to $p\cdot g=q\cdot h$.
  Since there is no torsion this action will give us a vector space.

  Now suppose that $\rH_i(X)$ is a $\bb{Q}$ vector space. Then there is no torsion in
  $\rH_i(X)$ as otherwise it wouldn't be a vector space. It then follows that $\rH(X;\bb{Z}_p)=0$
  as both the left and right groups of the short exact sequence will be forced to be zero.
\end{proof}

\sk

\begin{problem}[2.2.43(a)]
  Show that a chain complex of a free abelian groups $C_n$ splits as a
  direct sum of subcomplexes $0\rightarrow L_{n+1}\rightarrow K_n\rightarrow 0$
  with at most two nonzero terms. [Show that the short exact sequence
  $0\rightarrow\ker\partial\rightarrow C_n\rightarrow\text{im\ }\partial\rightarrow 0$
  splits and take $K_n=\ker\partial$.]
\end{problem}

\begin{proof}
  First note that the short exact sequence
  \[
    \xymatrix{
      0\ar[r] & \ker\partial_n\ar[r] & C_n \ar[r] & \text{im\ }\partial_{n+1} \ar[r] & 0
    }
  \]
  splits into
  \[
    \xymatrix{
      0\ar[r] & \ker\partial_n\ar[r] & \ker\partial_n\oplus\text{im\ }\partial_{n+1} \ar[r] & \text{im\ }\partial_{n+1} \ar[r] & 0
    }
  \]
  since $\text{im\ }\partial_{n+1}$ is free. Let $K_n:=\ker\partial_n$ and let $L_n:=\text{im\ }\partial_{n+1}$.
  The sequence $0\rightarrow L_{n+1}\rightarrow K_n\rightarrow 0$ is a chain complex as $\partial^2=0$. It then follows that we
  can express the chain complex
  \[
    \xymatrix{
      \cdots \ar[r] & C_{n+1} \ar[r] & C_n \ar[r] & C_{n+1} \ar[r] & \cdots
    }
  \]
  as a direct sum of the form
  \[
    \xymatrix{
      & & 0 \ar[r] & L_{n-1} \ar[r] & \cdots\\
      & 0 \ar[r] & L_n \ar[r] & K_{n-1} \ar[r] & 0\\
      0 \ar[r] & L_{n+1} \ar[r] & K_n \ar[r] & 0 &\\
      \cdots \ar[r] & K_{n+1} \ar[r] & 0 & &\\
    }
  \]
\end{proof}

\sk

\begin{problem}[2.3.4]
  Show that the wedge axiom for homology theories follows from the other
  axioms in the case of finite wedge sums.
\end{problem}

\begin{proof}
  From Hatcher we have that the Mayer-Vietoris sequence holds, even without
  the wedge axiom. If we calculate the wedge of two spaces $X_1$ and $X_2$, then
  we can deduce their homology using
  Mayer-Vietoris sequence along with the homology of the point being null to
  get
  \[
    \xymatrix{
      0 \ar[r] & h_i(X)\oplus h_i(Y) \ar[r] & h_i(X_1\vee X_2) \ar[r] & 0
    }
  \]
  Now assume that the wedge axiom holds when we wedge $n$ spaces together. As before
  if we calculate the homology of $\bigvee_1^{n+1}X_{i}$ we get
  \[
    \xymatrix{
      0 \ar[r] & h_i(\bigvee_1^n X_i)\oplus h_i(X_{n+1})\cong\bigoplus_1^n h_i(X_i) \ar[r] & h_i(\bigvee_1^{n+1}X_i) \ar[r] & 0
    }
  \]

  Therefore the wedge axiom follows from the other homology axioms when restricted
  to finite wedge sums.
\end{proof}

\sk

\begin{problem}[2.B.5]
  Let $S$ be an embedded $k$-sphere in $S^n$ for which there exists a disk
  $D^n\subset S^n$ intersecting $S$ in the disk $D^k\subset D^n$ defined by the
  first $k$ coordinates of $D^n$. Let $D^{n-k}\subset D^n$ be the disk defined
  by the last $n-k$ coordinates, with boundary sphere $S^{n-k-1}$. Show that the
  inclusions $S^{n-k-1}\hookrightarrow S^n\setminus S$ induces an isomorphism on
  homology groups.
\end{problem}

\begin{proof}
  
\end{proof}

\sk

\begin{problem}[2.B.10]
  Use the transfer sequence for the covering $S^\infty\rightarrow\bb{R}P^\infty$
  to compute  $H_n(\bb{R}P^\infty;\bb{Z}_2)$.
\end{problem}

\begin{proof}
  First recall that the infinite sphere $S^\infty$ has zero homology except
  for zeroth homology. It then follows if we apply the transfer sequence
  for $i>1$
  \[
    \xymatrix{
      0 \ar[r] & H_i(\bb{R}P^\infty;\bb{Z}_2) \ar[r] & H_{i-1}(\bb{R}P^\infty;\bb{Z}_2) \ar[r] & 0
    }
  \]
  we see that $H_i(\bb{R}P^\infty;\bb{Z}_2)\cong H_{i-1}(\bb{R}P^\infty;\bb{Z}_2)$ for $i>1$.
  We know that $H_0(\bb{R}P^\infty;\bb{Z}_2)\cong \bb{Z}_2$ since there only a single
  connected component. As such all we need to figure out is $H_1(\bb{R}P^\infty;\bb{Z}_2)$
  and this will determine the rest.

  Using the transfer sequence at the tail end we get
  \[
    \xymatrix{
      0 \ar[r] & H_i(\bb{R}P^\infty;\bb{Z}_2) \ar[r]^{\partial_*} & H_0(\bb{R}P^\infty;\bb{Z}_2)\cong\bb{Z}_2 \ar[r]^{\tau_*} & H_0(S^\infty;\bb{Z}_2)\cong\bb{Z}_2 \ar[r]^{p_*} & H_0(\bb{R}P^\infty;\bb{Z}_2)\cong\bb{Z}_2 \ar[r] & 0
    }
  \]
  First note that $p_*$ will be an isomorphism as it a surjective map between
  finite groups. This makes $\tau_*$ the zero map making $\partial_*$ surjective.
  However $\partial_*$ is already injective by exactness. So $\partial_*$ is an isomorphism
  and thus $H_1(\bb{R}P^\infty;\bb{Z}_2)=\bb{Z}_2$. Using our isomorphisms for the
  higher homology groups we get
  \[
    H_i(\bb{R}P^\infty;\bb{Z}_2) = \bb{Z}_2
  \]
  for all $i\geq 0$.
\end{proof}

%%%%%%%%%%%%%%%%%%%%%%%%%%%%%%%%%%%%%%%%%%%%%%%%%%%%%%%%%%%%%%%%%%%%%%%%%%%%%
\end{document}
