\documentclass[10pt]{article}
\usepackage[utf8]{inputenc}
\usepackage{amscd}
\usepackage{amsmath}
\usepackage{amssymb}
\usepackage{amsthm}
\usepackage{listings}
\usepackage{enumerate}
\usepackage{mathtools}
\usepackage[all,cmtip]{xy}

\textwidth=15cm \textheight=22cm \topmargin=0.5cm \oddsidemargin=0.5cm \evensidemargin=0.5cm

\newcommand{\sk}{\vskip 10mm}
\newcommand{\bb}[1]{\mathbb{#1}}
\newcommand{\ra}{\rightarrow}
\newcommand{\rH}{\widetilde{H}}

\theoremstyle{plain}
\newtheorem{problem}{Problem}
\newtheorem{lemma}{Lemma}[problem]

\theoremstyle{remark}
\newtheorem{tpart}{}[problem]
\newtheorem*{ppart}{}

\begin{document}


\begin{problem}[2.1.17]
  \begin{enumerate}
  \item[(a)] Compute the homology groups $H_n(X,A)$ when $X$ is $S^2$ or
    $S^1\times S^1$ and $A$ is a finite set of points in $X$.
  \item[(b)] Compute the groups $H_n(X,A)$ and $H_n(X,B)$ for $X$ a
    closed orientable surface of genus two with $A$ and $B$ the circles
    shown ($A$ is circle around join of Tori and $B$ is around tube of right
    tori). [What are $X/A$ and $X/B$.]
  \end{enumerate}
\end{problem}

\begin{proof}
  
\end{proof}

\sk

\begin{problem}[2.2.40]
  From the long exact sequence of homology groups associated to the short
  exact sequence of chain complexes
  $0\rightarrow C_i(X)\xrightarrow{n} C_i(X)\rightarrow C_i(X;\bb{Z}_n)\rightarrow 0$
  deduce immediately that there are short exact sequences
  \[
    0\rightarrow H_i(X)/nH_i(X)\rightarrow H_i(X;\bb{Z}_n)\rightarrow n-Torsion(H_{i-1}(X))\rightarrow 0
  \]
  where $n-Torsion(G)$ is the kernel of the map $G\xrightarrow{n} G$,
  $g\mapsto ng$. Use this to show that $\rH_i(X;\bb{Z}_p)=0$ for all $i$ and all
  primes $p$ iff $\rH_i(X)$ is a vector space over $\bb{Q}$ for all $i$.
\end{problem}

\begin{proof}
  
\end{proof}

\sk

\begin{problem}[2.2.43(a)]
  Show that a chain complex of a free abelian groups $C_n$ splits as a
  direct sum of subcomplexes $0\rightarrow L_{n+1}\rightarrow K_n\rightarrow 0$
  with at most two nonzero terms. [Show that the short exact sequence
  $0\rightarrow\ker\partial\rightarrow C_n\rightarrow\text{im\ }\partial\rightarrow 0$
  splits and take $K_n=\ker\partial$.]
\end{problem}

\begin{proof}
  
\end{proof}

\sk

\begin{problem}[2.3.4]
  Show that the wedge axiom for homology theories follows from the other
  axioms in the case of finite wedge sums.
\end{problem}

\begin{proof}
  
\end{proof}

\sk

\begin{problem}[2.B.5]
  Let $S$ be an embedded $k$-sphere in $S^n$ for which there exists a disk
  $D^n\subset S^n$ intersecting $S$ in the disk $D^k\subset D^n$ defined by the
  first $k$ coordinates of $D^n$. Let $D^{n-k}\subset D^n$ be the disk defined
  by the last $n-k$ coordinates, with boundary sphere $S^{n-k-1}$. Show that the
  inclusions $S^{n-k-1}\hookrightarrow S^n\setminus S$ induces an isomorphism on
  homology groups.
\end{problem}

\begin{proof}
  
\end{proof}

\sk

\begin{problem}[2.B.10]
  Use the transfer sequence for the covering $S^\infty\rightarrow\bb{R}P^\infty$
  to compute  $H_n(\bb{R}P^\infty;\bb{Z}_2)$.
\end{problem}

\begin{proof}
  
\end{proof}

%%%%%%%%%%%%%%%%%%%%%%%%%%%%%%%%%%%%%%%%%%%%%%%%%%%%%%%%%%%%%%%%%%%%%%%%%%%%%
\end{document}
